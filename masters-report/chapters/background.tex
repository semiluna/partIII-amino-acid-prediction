{
    \color{red}
    \begin{enumerate}
        \item Biological background
        \begin{enumerate}
            \item residue identity prediction
            \item atomic environments 
            \item SMILES representation?
        \end{enumerate}
        \item Machine learning background
        \begin{enumerate}
            \item neural networks
            \item graph neural networks
            \item equivariant graph neural networks
            \item representing residues as graphs
        \end{enumerate}
    \end{enumerate}
}

\section{Biological background}
\subsection{Proteins}
\begin{figure}
    \centering
    \begin{tikzpicture}
    \node (acidone) {\chemfig{N(-[3]H)(-[5]H)-C(-[2]H)(-[6]R_1)-C(-[1]{\color{blue}OH})(=[7]O)}};
    \node[right of=acidone, xshift=5cm] (acidtwo) {\chemfig{N(-[3]H)(-[5]{\color{blue}H})-C(-[2]H)(-[6]R_2)-C(-[1]OH)(=[7]O)}};
    \draw[->,thick] ($(acidone.east)!0.5!(acidtwo.west)$) ++(0, -1.7cm) -- ++(0,-1cm) node[right] {};
    \node[below of=acidone, xshift=3cm, yshift=-4.2cm] (residue) 
        {\chemfig{N(-[3]H)(-[5]H)-C(-[2]H)(-[6]R_1)-C(=[7]O)-[1,,,,blue]N(-[3]H)-C(-[2]H)(-[6]R_2)-C(=[7]O)(-[1]OH)}};
    
    \node[below of=residue, yshift=-1.3cm, xshift=-1.8cm] (res1) {\textit{residue 1}};
    \node[below of=residue, yshift=-1.3cm, xshift=1.8cm] (res1) {\textit{residue 2}};
    \end{tikzpicture}
\end{figure}
\subsection{Residue identity prediction}

\section{Machine learning background}
\subsection{Graph Neural Networks}
\subsection{Equivariant Graph Neural Networks}
\subsection{Representing residues as graphs}
